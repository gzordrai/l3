\documentclass{article}
\usepackage{listings}

\begin{document}
    Haskell est uniquement basé sur des expressions
    On ne controle pas le flow d'execution

    \begin{lstlisting}
        f n = x^2 + 2 * x * y - y^2
            where
                x = n^2
                y = 3 * n

        f' n = let
                x = n^2
                y = 3 * n
            in n^2 + 2 * x * y + y^2
    \end{lstlisting}

    Le typage:
        \begin{itemize}
            \item unit ()
                \begin{itemize}
                    \item Type qui contient qu'une seule valeu
                    \item Calculer la valeur de () c'est toujours calculer ()
                    \item Equivalent de void
                    \item Tuple de dimension 0
                \end{itemize}
            \item bool
                \begin{itemize}
                    \item True ou False
                    \item &&
                    \item ||
                    \item not
                \end{itemize}

        \end{itemize}

    Les gardes:
    \begin{lstlisting}
        h n = if n < 0 then 0
            else if n == 0
            then 1
            else n - 1

        h' n =  | n < 0 = 0
                | n == 0 = 1
                | otherwise = n - 1
    \end{lstlisting}

    L'ordre des gardes (|) est prit en compte

    Filtrage de motif:
    \begin{lstlisting}
        condMinus _ 0 = 0
        condMinus True n    | n < 0 = -n
                            | n > 0 = n
        condMinus False n   | n < 0 = n
    \end{lstlisting}

    L'opérateur ->
    \begin{itemize}
        \item Une fonction a un type domaine a et un type image b
        \item le type d'une telle fonction se note a -> b
        \item Il est d'usage de donner son type a une fonction que l'on déclare
        \begin{lstlisting}
            double :: Int -> Int
        \end{lstlisting}
    \end{itemize}

    Quel est le type de \f g e -> f e (g e)
    \begin{lstlisting}
        \f g e -> f e (g e) :: (c -> b -> a) -> (c -> b) -> c -> a
    \end{lstlisting}
    f => operation
    e => environement d'un code (bloque mémoire)
    g => un code 
    Cette lamda fonction est beaucoup utilisé pour la compilation pour appliqué une fonction a un bout de code avec le même environement que le code
\end{document}