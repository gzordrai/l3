\documentclass{article}
\usepackage{listings}

\title{Programmation fonctionelle}
\begin{document}
    La programmation fonctionnelle est un paradigme de programmation (comme le paradigme objet) où les fonctions jouent un rôle central:

    \begin{itemize}
        \item dans la modélisation
        \item dans le non duplication de code
        \item dans le contrôle de l'éxécution des programmes
    \end{itemize}

    Historiquement les premiers modèles de calcul ont été basés sur les ? tion de fonction mathématique: fonction primitives récursives, fonction générales récursives,
    lambda-calcul (A Church)

    Les premiers langages de programmation se sont inspirés des travaux de Church (1950) par exemple est un langage de fonctionnel.

    En général  ce que l'on appelle fonction de dans un langage de programmation difère de la notion de fonction mathématique car 2 appels identiques à la même fonction
    peut donner des résultats différents

    Exemple en python:
    \begin{lstlisting}[language=Python]
        x = 1

        def plus_x(k):
            global x
            x += 1

        return k + x
    \end{lstlisting}

    Ici plus\_x(0) renvoie 2 au premier apppel et 3 au second.

    Dans les langages fonctionneles les fonctions diffèrent souvent de la notion de fonction mathématiques
    \begin{itemize}
        \item Ocaml
        \item Lisp
        \item Schimc
        \item Erlang
        \item etc.
    \end{itemize}

    Nous allons étudier le langage Haskell qui:
    \begin{itemize}
        \item est fonctionelle pur c'est à dire les fonctions sont très "proches" des fonctions mathématiquesx = 1
        \item est parresseux ce qui implique que l'on peut multiplier des ? potentiellement infinies
        \item difficile à apprendre
        \item syntaxe particulièrement élégante
        \item est fortement typé: donne des garanties à l'éxécution

        \begin{itemize}
            \item permet des fortes optimisation de code
            \item rejette les programmes mal construits
        \end{itemize}

        \item est à la source de nb idées que st dans des langages récents: clotures, monades, optimisation
    \end{itemize}

    Objectif de l'UE: écrire des programmes correct, clair, concis, réutilisable en Haskellet maitriser les concepts sous ?

    Exezmples de programmes Haskell:
    Faire la somme des n 1er nombre

    \begin{lstlisting}[language=Haskell]
        somme n = sum [1 .. n]
    \end{lstlisting}

    somme   = nom de la fonction
    n       = argument de la fonction
    sum     = fonction librairie standard qui fait la somme des arguments des éléments de la liste d'argument
    ..      = fonction d'énumération qui énumère tous les éléments entre ses 2 args

    Faire un quick sort
    \begin{lstlisting}[language=Haskell]
        qsrot [] = []
        qsort (x : xs) = qsort petits ++ [x] ++ qsort grands
            where
                petits = [y | y <- xs, y <= x]
                grands = [y | y <- xs, x < y]
    \end{lstlisting}

    Faire la somme des n 1er nombre en récursif
    \begin{lstlisting}
        sommeRec n = if n == 0
                    then 0
                    else n + sommeRec (n - 1)
    \end{lstlisting}
\end{document}